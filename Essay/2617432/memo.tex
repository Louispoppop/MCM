\section{Memo}

\noindent
\textbf{To:} The Producers of \textit{Dancing with the Stars} \\
\textbf{From:} MCM Team 2617432 \\
\textbf{Date:} February 2, 2026 \\
\textbf{Subject:} Suggestions for Improving the Rating System

\par\noindent\rule{\textwidth}{0.2pt}

\noindent Dear Producers,

\noindent
\begin{minipage}[t]{0.65\textwidth}
    \setlength{\parindent}{2em}
    \indent We've analyzed 34 seasons of data from \textit{Dancing with the Stars}, trying to understand how different rating combination rules affect the competition results. Now, we've come to some conclusions and suggestions that we believe can help make the show even better in the future.
\end{minipage}
\hfill
\begin{minipage}[t]{0.3\textwidth} % 右侧占据 30% 的宽度
    \vspace{-1cm} % 稍微向上微调图片位置
    \raggedleft % 图片靠右对齐
    \includegraphics[width=3cm]{logo_new.png}
\end{minipage}


\subsubsection*{What We Found from Past Rules}

We compared the two main systems you have used:

\begin{itemize}
    \item \textbf{Percentage System:} This method clearly reflects fan passion but carries high risk. Extremely popular contestants can easily overpower judge scores, making dancing skill less important.
    \item \textbf{Rank System:} This method is fairer but partially ignores fans' support. A lead of one million votes counts the same as a single vote, which will dampen fan enthusiasm.
    \item \textbf{Judges' Choice:} Allowing judges to save one of the bottom two couples is highly effective. It prevents talented dancers from being eliminated due to voting anomalies.
\end{itemize}

\subsubsection*{Our Proposal: The Adaptive System}

To seek for better solutione, we designed the \textbf{Adaptive Logistic Meritocracy System} (ALMS) to balance fairness and popularity without compromise.

\textbf{1. Limiting Extreme Monopolies:}
The system counts fan votes fully up to a reasonable point, then gradually reduces the weight of excess votes. This prevents a single contestant from breaking the game mechanics while still rewarding popularity.

\textbf{2. Dynamic Sensitivity:}
The system automatically adjusts based on competition tightness. It becomes more sensitive to small differences when the race is close to ensure the best dancer wins.

Finally, our tests show that this system improves fairness by 13.3\% compared to the Percentage rule while retaining 95.4\% of fan satisfaction. We suggest using this model for future seasons, for it truly maintains the excitement of voting while protecting the competitive nature of the show.

Besides, retaining the Judges' Save is a good idea to ensure technical integrity. You should also focus on "the Middle": Voting impact is highest for mid-tier contestants. Highlighting these close battles will drive more meaningful engagement than focusing on safe leaders.

We hope our analysis provides a fresh and useful perspective for your team.
