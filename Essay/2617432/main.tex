\documentclass{mcmthesis}  % 当使用 CTeX 套装时请注释上一行使用该行的设置
\mcmsetup{tstyle=\color{red}\bfseries,%修改题号,队号的颜色和加粗显示,黑色可以修改为 black
tcn = 2617432, problem = C, %修改队号,参赛题号
sheet = true, titleinsheet = true, keywordsinsheet = true,
titlepage = false, abstract = true}
\usepackage{float}
\usepackage{subcaption}
\usepackage{algorithm}
\usepackage{algorithmicx}
\usepackage{algpseudocode}
\usepackage{tocloft}
\usepackage{enumitem}

\tocloftpagestyle{fancy}
\setlength{\cftbeforesecskip}{2pt}      % section 前间距
\setlength{\cftbeforesubsecskip}{1pt}   % subsection 前间距
\setlength{\cftparskip}{0pt}  
\setlist[itemize]{topsep=0pt, parsep=0pt, partopsep=0pt} % 全局去除 itemize 顶部间距
\setlist[enumerate]{topsep=0pt, parsep=0pt, partopsep=0pt} % 全局去除 enumerate 顶部间距

%四款字体可以选择
\usepackage{times}
%\usepackage{newtxtext}
%\usepackage{palatino}
%\usepackage{txfonts}

\usepackage[utf8]{inputenc}
\usepackage{indentfirst}  %首行缩进,注释掉,首行就不再缩进。
\usepackage{lipsum}

%\usepackage{ctex}

% 调整页眉高度,消除 fancyhdr 的 \headheight too small 警告
\setlength{\headheight}{14pt}

\title{Popularity vs. Performance: Evaluation and Adaptive Optimization of Rating Mechanisms Based on DWTS}

\begin{document}
\begin{abstract}

    For thirty-four seasons, \textit{Dancing with the Stars} has captivated audiences by blending ballroom excellence with celebrity charisma. However, the mechanism combining professional judge scores and public fan votes leads to some loopholes, historically sparking controversy when popular contestants survive despite poor technical performance. Our research seeks to deeply explore the original mechanism of DWTS and propose a better-balanced model.

    \textbf{Fan Voting Distribution Estimation:} To overcome the challenge of unobservable fan data, we developed an \textbf{Inverse Bayesian Estimation Model}. By treating historical elimination outcomes as strict constraints on a Dirichlet distribution, we employed Monte Carlo simulations to infer the latent voting shares for every week, during which the assumed base popularity of contestants are calculated. The model achieved a \textbf{98.3\%} historical reproduction rate and revealed a "Bubble Effect", where voting certainty is highest for at-risk contestants.

    \textbf{Analysis of Celebrity Characteristics:} We employed a \textbf{Linear Mixed-Effects Model (LMM)} to analyze key characteristics by fitting three target variables: final placement, judge scores, and fan votes. Notably, the model for fan voting achieved an exceptional $R^2$ of \textbf{0.989}. We concluded that \textbf{Base Popularity} and \textbf{Professional Partner Strength} are the primary determinants of competition outcomes. Comparative analysis further reveals a distinct divergence: judges evaluate a multi-dimensional matrix of technical factors, whereas fan support is overwhelmingly driven by pre-existing popularity rather than performance quality.

    \textbf{Benchmark Rule Evaluation and Judges' Choice Mechanism:} We established a \textbf{Dual-Track Framework} to compare the show's two historical systems: Rank-based and Percent-based. Simulations confirmed while the Rank system prevents monopoly, it suppresses fan enthusiasm. Conversely, the Percent system suffers from "Linear Vulnerability".

    \textbf{The creation of the well-balanced Mechanism:} We propose the \textbf{Adaptive Logistic Meritocracy System (ALMS)}. This novel mechanism introduces a Sigmoid saturation curve to impose a "soft ceiling" on extreme popularity and dynamically adjusts sensitivity based on competition intensity. Stress-testing demonstrates that ALMS achieves a Pareto improvement. \textbf{Sensitivity analysis} on both the Bayesian inference hyperparameters and the ALMS control parameters validates the robustness of our conclusions across a broad parameter space. We recommend ALMS, alongside the "Judges' Choice," as the optimal strategy for the show's post-game era.

    \begin{keywords}
        \textbf{Inverse Bayesian Inference, Monte Carlo Simulation, Linear Mixed-Effects Model, Mechanism Evaluation, Adaptive System, Dancing with the Stars}
    \end{keywords}

\end{abstract}

\maketitle
\setcounter{tocdepth}{2}
\tableofcontents

\newpage
\section{Introduction}

\subsection{Background}

"Dancing with the Stars" (DWTS) is a globally recognized reality competition where celebrity and professional dance pairs are evaluated through a hybrid voting system combining expert judge scores and public fan votes. The core mechanic of the show relies on aggregating two distinct data streams: the objective, technical evaluation of judges (1-10 scale) and the subjective, popularity-driven support of the audience.

\begin{figure}[H]
    \centering
    \includegraphics[width=0.5\linewidth]{figures/introduction.jpg}
    \caption{The Post of the Show}
    \label{fig:intro}
\end{figure}

Throughout its 34-season history, the U.S. version of DWTS has utilized different aggregation methods—specifically, Rank-based and Percentage-based combinations—to determine eliminations. Discrepancies between technical merit and popularity have occasionally led to "controversial" outcomes, where contestants with low judge scores survive or even win due to overwhelming fan support (e.g., Jerry Rice in Season 2, Bobby Bones in Season 27). These anomalies prompted production changes, including the "Judges' Save" introduced in Season 28. Understanding the mathematical implications of these voting mechanisms is crucial for balancing fairness (skill) and viewer engagement (popularity).

\subsection{Literature Review}

The problem of combining expert scores with public voting falls within the intersection of Social Choice Theory and Statistical Estimation.

\begin{itemize}
    \item Social Choice Theory: Traditional voting literature, prominently Arrow’s Impossibility Theorem, suggests that no rank-order voting system is perfectly fair under all criteria. Methods like the Borda Count (similar to the Rank-based method) are often contrasted with cardinal utility methods (similar to the Percentage-based method).
    \item Latent Variable Models: To estimate unknown fan votes from binary outcomes (eliminated/safe), we draw upon literature regarding Inverse Problems and Logistic Regression. These methods allow for the inference of hidden parameters that best explain the observed state transitions.
    \item Performance Analytics: Previous studies on competition dynamics suggest that "star power" and demographic relatability often outweigh technical skill in public polling, a phenomenon we will analyze using multivariate regression techniques.
\end{itemize}

\subsection{Clarifications and Restatements}

The central challenge of this study is to analyze the mechanics of the DWTS voting system and the interactions between expert evaluation and public opinion. Specifically, the observed data includes judge scores and elimination results, but the actual volume of fan votes remains a latent (hidden) variable. To address this, we have broken down the problem into four specific tasks:

Task 1: Fan Vote Estimation
We must develop a mathematical model to reconstruct the unobserved fan votes for each contestant. This involves validating the model by ensuring the estimated votes, when combined with judge scores, consistently reproduce the historical weekly elimination results. We must also quantify the certainty of these estimates.

Task 2: Comparative Analysis of Voting Mechanisms
Using the estimated fan votes, we will apply both the Rank-based and Percentage-based methods to historical data across seasons. This analysis aims to determine which method favors popularity over technical skill and to re-examine specific controversial outcomes (e.g., Season 27). We will also assess the impact of the "Judges' Save" mechanism.

Task 3: Factor Analysis of Performance
We are required to model the influence of external factors—including professional partners, celebrity demographics (age, industry), and geographic background—on both the celebrities' final placement and the distinct components of their score (judge evaluations vs. fan support).

Task 4: System Optimization
Finally, we will propose a novel voting framework designed to optimize the trade-off between fairness and entertainment value, providing specific recommendations for future seasons to the show's producers.

\subsection{Our Work}

Briefly outline your main ideas and contributions. A typical structure is:
\begin{itemize}
  \item Propose a mathematical model (or several models) that captures the key mechanisms of the problem;
  \item Design numerical algorithms or solution procedures to solve the model efficiently;
  \item Carry out experiments or simulations and compare with baseline methods;
  \item Provide sensitivity analysis, evaluation, and possible extensions.
\end{itemize}


\newpage
\section{Assumptions}

% In this section, list and explain all modeling assumptions you make. Each assumption should be realistic, necessary, and clearly stated.

\subsection{Task 1}

\begin{description}
  \item[Assumption 1: Performance-Popularity Mixture Hypothesis] Fan voting is driven by a dual mechanism: a long-term \textit{Base Popularity} (invariant fan base) and a short-term \textit{Performance} component (immediate reaction to the dance). We model the expected vote share as a linear combination of these two factors.
  \item[Assumption 2: Rational Judges' Save (Season 28+).] For seasons where judges choose between the bottom two couples, we simplify the model by assuming judges consistently save the couple with the higher technical score for that week.
  \item[Assumption 3: Dirichlet Distribution for Vote Shares.] Since vote shares must sum to 1, the Dirichlet distribution is the natural conjugate prior for multinomial distributions on a simplex, providing a flexible framework to model variances in voting proportions.
  \item[Assumption 4: Constant Total Vote Volume.] Although viewership fluctuates, the shift from telephone to app-based voting has stabilized engagement. Based on recent data, we anchor the total weekly votes at a constant $V_{total} = 14$ million to facilitate cross-week comparisons.
\end{description}


% You can add, delete, or refine the above assumptions according to the specific contest problem.


\newpage
\section{Notations}
\setlength{\parindent}{2em}

The core symbols and their definitions used in this study are summarized in Table~\ref{table:notations}, providing an overview of the key parameters and their related meanings.

\begin{table}[ht]
  \centering
  \caption{\textbf{Notations used in this paper}}
  \renewcommand{\arraystretch}{1.5}
  \begin{tabular}{>{\centering\arraybackslash}p{0.3\linewidth} >{\centering\arraybackslash}p{0.6\linewidth}}
    \toprule
    \textbf{Symbol} & \textbf{Description} \\
    \midrule
    $\text{CSY}$  & Fisherman \\
    \bottomrule
  \end{tabular}
  \label{table:notations}
\end{table}

\begin{table}[ht]
  \centering
  \caption{Main variables and meanings}
  \renewcommand{\arraystretch}{1.2}
  \begin{tabular}{cll}
    \toprule
   \textbf{Symbol}&\textbf{Description}&\textbf{Definition / Value}\\
    \midrule
        $\mathbf{S}_t$       & Latent fan vote share vector at week $t$           & $\mathbf{S}_t \in \Delta^{n_t-1}$                            \\
        $\mathbf{J}_t$       & Observed judges' score vector at week $t$          & $\mathbf{J}_t \in \mathbb{R}^{n_t}$                          \\
        $\mathcal{O}_t$      & Observed elimination outcome at week $t$           & Binary / Categorical                                         \\
        $\boldsymbol{\beta}$ & Latent Base Popularity vector                      & $\boldsymbol{\beta} \in \Delta^{N-1}$                        \\
        $\mathbf{P}_t$       & Normalized judge performance vector                & $P_{i,t} = J_{i,t} / \sum J_{k,t}$                           \\
        $\boldsymbol{\mu}_t$ & Expected voting propensity                         & $\mu_{i,t} = E[S_{i,t}]$                                     \\
        $\lambda$            & Performance weight coefficient                     & Constant ($0.2$)                                             \\
        $\kappa$             & Dirichlet concentration parameter (Prior strength) & Constant ($50.0$)                                            \\
        $\text{Ext}_t$       & Set of active contestants in week $t$              & Subset of $\mathcal{C}$                                      \\
        $\Phi_t$             & Set of valid samples satisfying constraints        & $\Phi_t = \{\mathbf{s} \mid L(\mathcal{O}_t|\mathbf{s})=1\}$ \\
        $CV_{i,t}$           & Coefficient of Variation for partial uncertainty   & $CV = \sigma / \mu$\\
        \bottomrule
    \end{tabular}
\label{table:variables}
\end{table}



\newpage
% In this section, build the core mathematical models for the problem.
\section{Model I: Bayesian Estimation of Fan Voting Distribution}

\subsection{Problem Formulation}
The estimation of spectator voting data poses
an \textbf{Inverse Problem} involving multi-dimensional hidden variables.
Our objective is to infer the unobserved fan vote shares $\mathbf{S}_t$
given the observed judges' scores $\mathbf{J}_t$
and the corresponding elimination outcomes $\mathcal{O}_t$.

The target variable resides on the standard simplex:
\begin{equation}
    \Delta^{n_t-1} = \left\{ \mathbf{x} \in \mathbb{R}^{n_t} \bigg| \sum_{i=1}^{n_t} x_i = 1, x_i > 0 \right\}
\end{equation}

Analytical derivation of the posterior
$p(\mathbf{S}_t | \mathbf{J}_t, \mathcal{O}_t)$ is difficult
due to the discontinuous and non-differentiable nature of the aggregation rules
(switching between Rank-based and Percentage-based systems).
To tackle this problem, we adopt a \textbf{Simulation-Based Inference} framework,
using Approximate Bayesian Computation (ABC)
to approximate posterior probabilities
and perform inference by comparing millions of random scenarios
with historical realities, i.e., the Monte Carlo method.

\subsection{Mathematical Model Establishment}

\subsubsection{The Mixture Mean Hypothesis}
We assume that the \textit{Expected Propensity} $\boldsymbol{\mu}_t$
of fan votes is a weighted combination of two distinct forces:
the contestant's \textit{Base Popularity}
$\boldsymbol{\beta}$ (long-term fan loyalty, which is time-invariant in each season and latent)
and their \textit{Judge Performance} $\mathbf{P}_t$ (short-term dance quality).
\begin{equation}
    \mu_{i,t} = (1 - \lambda) \cdot \frac{\beta_i}{\sum_{k \in \text{Ext}_t} \beta_k} + \lambda \cdot P_{i,t}
    \label{eq:mixture}
\end{equation}

Here, $\lambda$ represents the weight of dance performance in the fans' decision.
Based on the controversial elimination
in the early seasons of this show, we set $\lambda=0.4$. This reflects the fact that DWTS audiences care more about their favorite celebrities' popularity than on-stage performance. 

The actual votes follow a \textbf{Dirichlet distribution}:
$\mathbf{S}_t \sim \text{Dirichlet}(\kappa \cdot \boldsymbol{\mu}_t)$,
where $\kappa=50$ controls the variance to keep the distribution stable and realistic.

\subsubsection{Constraints from Elimination Results}
The historical elimination results act as strict rules.
We can define the Likelihood function $L(\mathcal{O}_t | \mathbf{S}_t, \mathbf{J}_t)$
as a simple filter.
It equals 1 if the simulated votes lead to the correct elimination outcome $\mathcal{O}_t$, and 0 otherwise:
\begin{itemize}
    \item \textbf{Rank Rule:} The contestant with the lowest sum of $\text{Rank}(\mathbf{S}_t) + \text{Rank}(\mathbf{J}_t)$ is eliminated.
    \item \textbf{Percentage Rule:} The contestant with the lowest total score $\mathbf{S}_t + \mathbf{P}_t$ is eliminated.
    \item \textbf{Judges' Choice:} The elimination is valid if the actual eliminated contestant falls within the \textbf{Bottom-2} (the risk zone), making them eligible for elimination.
\end{itemize}

\subsection{Algorithm: Iterative Inverse Estimator}

To find the unknown base popularity $\boldsymbol{\beta}$, we designed an iterative algorithm. It works similarly to the \textbf{Expectation-Maximization (EM)} method.

\begin{algorithm}[H]
    \caption{Iterative Solving for Base Popularity $\boldsymbol{\beta}$}
    \begin{algorithmic}[1]
        \State \textbf{Input:} Scores $\mathbf{J}$, Eliminations $\mathcal{O}$. \textbf{Output:} Base Popularity $\hat{\boldsymbol{\beta}}$
        \State \textbf{Initialize:} $\boldsymbol{\beta}^{(0)} \leftarrow [1/N, \dots, 1/N]$ (Equal start)
        \Repeat
        \State $k \leftarrow k+1$
        \For{each week $t = 1 \dots T$}
        \State \textbf{Sampling:} Generate $M$ vote scenarios $\{\mathbf{S}_t^{(m)}\}$ based on current popularity $\boldsymbol{\beta}^{(k)}$.
        \State \textbf{Filtering:} Keep only the scenarios that match the historical elimination result $\mathcal{O}_t$.
        \State \textbf{De-mixing:} Isolate the implied popularity $\hat{\boldsymbol{\beta}}_t$ by inverting the mixture equation (\ref{eq:mixture}):
        $$ \hat{\boldsymbol{\beta}}_t \leftarrow \frac{\bar{\mathbf{S}}_t - \lambda \mathbf{P}_t}{1 - \lambda} $$
        \Statex \hskip\algorithmicindent (Where $\bar{\mathbf{S}}_t$ is the average of valid scenarios)
        \EndFor
        \State \textbf{Update:} Update $\beta_i^{(k+1)}$ by averaging $\hat{\beta}_{i,t}$ across all weeks.
        \Until{Convergence of $\boldsymbol{\beta}$}
        \State \Return $\hat{\boldsymbol{\beta}}$
    \end{algorithmic}
    \label{alg:pop_solve}
\end{algorithm}

\begin{algorithm}[H]
    \caption{Monte Carlo Posterior Estimation}
    \begin{algorithmic}[1]
        \State \textbf{Input:} Optimized Popularity $\hat{\boldsymbol{\beta}}$, Scores $\mathbf{J}$, Eliminations $\mathcal{O}$
        \State \textbf{Output:} Estimated Votes $\hat{\mathbf{S}}$, Uncertainty $\boldsymbol{\sigma}$
        \For{each week $t = 1 \dots T$}
        \State \textbf{Proposal Step:} Generate $N=20,000$ random scenarios $\{\mathbf{S}_t^{(n)}\}$ from Dirichlet distribution centered on $\mu_t(\hat{\boldsymbol{\beta}})$.
        \State \textbf{Rejection Step:} Filter out valid scenarios $\Phi_t$:
        $$ \Phi_t = \{ \mathbf{s} \in \{\mathbf{S}_t^{(n)}\} \mid L(\mathcal{O}_t | \mathbf{s}) = 1 \} $$
        \State (Discard any scenario that contradicts the historical elimination)
        \If{$|\Phi_t| > 0$}
        \State $\hat{\mathbf{S}}_t \leftarrow \text{Mean}(\Phi_t)$ \Comment{Most likely vote distribution}
        \State $\boldsymbol{\sigma}_t \leftarrow \text{StdDev}(\Phi_t)$ \Comment{Standard Deviation as uncertainty}
        \Else
        \State $\hat{\mathbf{S}}_t \leftarrow \boldsymbol{\mu}_t$ \Comment{Fallback to prior for extreme outliers}
        \EndIf
        \EndFor
        \State \Return Sequence $\{\hat{\mathbf{S}}_t, \boldsymbol{\sigma}_t\}$ for all weeks.
    \end{algorithmic}
    \label{alg:monte_carlo}
\end{algorithm}

The main challenge is \textbf{Survivorship Bias}: sometimes, a dancer with low scores continues to survive. This implies they must have very high popularity. The algorithm \ref{alg:pop_solve} solves this by slowly increasing their popularity estimate in $\boldsymbol{\beta}$ until their survival makes mathematical sense.

With the optimized Base Popularity $\hat{\boldsymbol{\beta}}$ determined,
we proceed to the final phase: \textbf{Posterior Estimation}.
We employ a high-density Monte Carlo Sampling(Algorithm \ref{alg:monte_carlo})
to reconstruct the specific voting history and quantify uncertainty.


\subsection{Validation and Results}

\subsubsection{Consistency Verification}
The model achieves a \textbf{98.7\%} Historical Reproduction Rate across 34 seasons.
This means that in 98.7\% of weeks, our estimated fan votes correctly
characterize the actual eliminated contestant as the one with
the lowest combined score (or in the Bottom-2).
This high fidelity proves that our model is reliable.

\subsubsection{Uncertainty and The "Bubble Effect"}

To evaluate the reliability of our estimates,
we define the \textbf{Coefficient of Variation (CV)} as the metric for uncertainty: $CV = \sigma / \mu$.
A low CV implies high certainty. Our analysis,
particularly through the lens of a \textbf{Season 2 Case} Study,
reveals that prediction certainty is \textbf{not uniform}.

Moreover, we observe a phenomenon we term the \textbf{"Bubble Effect"}:
contestants at risk of elimination (on the "bubble")
show much higher estimation certainty than safe leaders.

\textbf{Evidence I: Uncertainty Evolution}

Figure \ref{fig:season2_cv} tracks the uncertainty throughout Season 2, in which Jerry Rice (at risk) shows consistently lower CV (higher certainty) than safe contestants like Lachey,validating the Bubble Effect. 

\begin{figure}[H]
    \centering
    \includegraphics[width=0.7\textwidth]{season2_cv_evolution.png}
    \caption{Uncertainty Evolution in Season 2}
    \label{fig:season2_cv}
\end{figure}

Since Rice was constantly near the elimination threshold, the mathematical constraints on his vote share were relatively tight; even small deviations would contradict his survival. Thus, the model locks in his popularity with high precision. In contrast, safe leaders operate under looser constraints, resulting in higher uncertainty.

\textbf{Evidence II: The Popularity Gap}

The root of Rice's "Bubble" status lies in the disparity between his skill and popularity (Figure \ref{fig:season2_pie}). 
During the finals, despite receiving the lowest judge scores (Left), 
Rice dominated the estimated fan vote with a 45.3\% share (Right). 
This massive imbalance kept him in the risky "Bubble" zone, while he is popular enough to survive, 
yet with scores low enough to remain vulnerable. 
This tension provides the strict constraints that allow our model to capture his data with high certainty.

\begin{table}[H]
    \centering
    \caption{Season 2 Finals Data Statistics and Analysis (Including Model Estimates)}
    \resizebox{\textwidth}{!}{%
        \begin{tabular}{lccccccccc}
            \hline
            \textbf{Celebrity} & \textbf{Score} & \textbf{J.Rank} & \textbf{Est.Share} & \textbf{Est.Votes} & \textbf{V.Rank} & \textbf{Sum} & \textbf{Result} & \textbf{Std} & \textbf{CV} \\ \hline
            Drew Lachey        & 29.00          & \#1              & 0.3259             & 4.56m              & \#2              & $1+2=3$      & 1st             & 0.0413       & 0.1266      \\
            Stacy Keibler      & 28.67          & \#2              & 0.2208             & 3.09m              & \#3              & $2+3=5$      & 3rd             & 0.0450       & 0.2036      \\
            Jerry Rice         & 26.67          & \textbf{\#3}     & \textbf{0.4533}    & 6.35m              & \textbf{\#1}     & $3+1=4$      & 2nd             & 0.0543       & 0.1198      \\ \hline
        \end{tabular}%
    }
    \label{tab:season2_results}
\end{table}

\begin{figure}[H]
    \centering
    \includegraphics[width=0.7\textwidth]{season2_final_pie.png}
    \caption{Season 2 Finals: Judge Score Distribution vs. Estimated Fan Vote Share.}
    \label{fig:season2_pie}
\end{figure}


This case study validates that our model not only reconstructs the voting history,
but also correctly reflects the mathematical properties of the elimination rules.

% Model II: Macro-Comparative Analysis of Aggragation Rules
\section{Model II: Macro-Comparative Analysis of Aggregation Rules}

\subsection{Problem Formulation and Simulation Framework}
Having reconstructed the latent fan voting distribution $\hat{\mathbf{S}}_t$ in Model I, we proceed to the second phase of our study: a rigorous, counterfactual evaluation of the competition's aggregation rules. Throughout the history of \textit{Dancing with the Stars}, two distinct scoring systems have been employed:
\begin{itemize}
    \item \textbf{Rank-based System (Scenario A):} Used in Seasons 1-2 and 28+. Utilizing the sum of ranks ($\min [R_J + R_F]$).
    \item \textbf{Percent-based System (Scenario B):} Used in Seasons 3-27. Utilizing the sum of shares ($\max [P_J + P_F]$).
\end{itemize}

The objective of Model II is to quantify the intrinsic bias of these rules. specifically, we ask: \textit{Which rule possesses a higher "Fan-Friendliness," i.e., a lower deviation from the pure popular vote?}

\subsubsection{Data Standardization and Dual-Track Simulation}
To conduct a fair cross-season comparison, we establish a **Dual-Track Counterfactual Simulation** framework. For every week $w$ in every season $s$, we construct two parallel universes, holding the contestants' performance (Judge Scores $\mathbf{J}$) and popularity (Estimated Fan Votes $\hat{\mathbf{S}}$) constant, while varying only the aggregation rule.

\textbf{Standardization for Comparability:}
Since "Ranks" and "Percentages" exist in different vector spaces, we map all percentages to the rank domain. Let $\mathbf{P}_{F,t}$ be the estimated fan vote share vector for week $t$. The implied Fan Rank $\mathbf{R}_{F,t}$ is derived as:
\begin{equation}
    \mathbf{R}_{F,t} = \text{Rank}(-\mathbf{P}_{F,t})
\end{equation}
where rank 1 corresponds to the highest vote share.

\textbf{Simulation Tracks:}
\begin{itemize}
    \item \textbf{Universe A (Mandatory Rank):} We compute the composite score $S_{A,i} = R_{J,i} + R_{F,i}$. The simulated final ranking $\mathbf{R}_{final, A}$ is the rank of these sums (ascending).
    \item \textbf{Universe B (Mandatory Percent):} We compute the composite score $S_{B,i} = P_{J,i} + P_{F,i}$. The simulated final ranking $\mathbf{R}_{final, B}$ is the rank of these sums (descending).
\end{itemize}

\subsection{Quantifying Bias: The Fan Deviation Index (FDI)}
We introduce the \textbf{Fan Deviation Index (FDI)} to measure the mechanism-induced distortion of the public will. FDI is defined as the normalized Manhattan distance between the rule-generated outcome and the pure fan preference.

For a specific rule $k \in \{Rank, Percent\}$ at week $w$ with $N_w$ contestants:
\begin{equation}
    \text{FDI}_k^{(w)} = \frac{1}{N_w} \sum_{i=1}^{N_w} \left| R_{final, k, i} - R_{F, i} \right|
\end{equation}

\begin{itemize}
    \item Low FDI ($\to 0$): The rule faithfully reflects fan voting (Fan-Friendly).
    \item High FDI: The rule allows judge scores to significantly override fan votes (Judge-Dominant).
\end{itemize}

\begin{figure}[H]
    \centering
    \includegraphics[width=0.8\linewidth]{model2_imgs/Task2_Subtask1_Boxplot.png}
    \caption{Distribution of Fan Deviation Index (FDI) for Rank-based and Percent-based rules. The Percent-based rule (Right) consistently shows lower FDI values, indicating it is mathematically more driven by fan voting variance.}
    \label{fig:fdi_boxplot}
\end{figure}

\begin{figure}[H]
    \centering
    \includegraphics[width=0.8\linewidth]{model2_imgs/Task2_Subtask1_BarChart.png}
    \caption{Weekly Comparison of Fan Deviation Index. In the majority of weeks, the Percent-based system yields a result closer to the pure fan vote ($\Delta > 0$).}
    \label{fig:fdi_barchart}
\end{figure}

\subsubsection{Results: The Variance Suppression Hypothesis}
We computed $\Delta = \text{FDI}_{Rank} - \text{FDI}_{Percent}$ across all 34 seasons.
\begin{itemize}
    \item \textbf{Observation:} In \textbf{83\%} of simulated weeks, $\Delta > 0$, implying $\text{FDI}_{Percent} < \text{FDI}_{Rank}$.
    \item \textbf{Interpretation:} The Percent-based system is structurally more "Fan-Friendly."
    \item \textbf{Mechanism:} This phenomenon arises from \textbf{Variance Mismatch}. Judge scores typically cluster in a narrow range (e.g., 7 to 9), whereas fan votes often exhibit extreme skew (e.g., one star getting 40\% while others get 5\%). In a summation $P_J + P_F$, the term with higher variance ($P_F$) mathematically dominates the sum. In contrast, the Rank system forces a uniform distribution (1, 2, ..., N) on both components, enforcing a strict 50-50 power sharing, which effectively "suppresses" the dominance of a super-popular celebrity.
\end{itemize}

\section{Sensitivity Analysis and Policy Optimization}

\subsection{Mechanism Stress Test: Analysis of "Controversial Survivors"}
To rigorously test the robustness of competition rules, we examine extreme edge cases—"Low-Score Survivors"—who historically sparked controversy by advancing despite poor technical scores. We selected four representative cases: Jerry Rice (S2), Billy Ray Cyrus (S4), Bristol Palin (S11), and Bobby Bones (S27).

We simulated their survival under four distinct regulatory combinations:
\begin{enumerate}
    \item \textbf{Rank Only:} Classic S1/S2 rules.
    \item \textbf{Percent Only:} Classic S3-S27 rules.
    \item \textbf{Rank + Judges' Save:} New S28+ rules (Bottom 2 veto).
    \item \textbf{Percent + Judges' Save:} Hypothetical hybrid.
\end{enumerate}

\textbf{Reversal Detection Logic:}
We define a \textbf{DANGER} state if a contestant who historically survived ($R_{actual} = \text{Safe}$) would have been eliminated under the simulated rule ($R_{sim} = \text{Eliminated}$).

\textbf{Key Findings: The "Rank + Save" Correction Effect}
The simulation results across all four controversial cases consistently point to the same mechanism of correction. As illustrated in Figure \ref{fig:combined_timelines}, the introduction of the \textbf{Rank + Judges' Save} rule effectively neutralizes the "popularity shield" that protected these low-scoring contestants in history.
\begin{itemize}
    \item \textbf{Early Intervention:} For Bristol Palin (S11) and Billy Ray Cyrus (S4), the hypothetical "Rank + Save" rule triggers a \textbf{DANGER} status (red block) significantly earlier than their actual elimination. This confirms that the mechanics of the Rank system, combined with a safety valve for skilled dancers, would have prevented these "unjust" advancements.
    \item \textbf{Preventing Anomalous Wins:} In the case of Bobby Bones (S27), who won under the Percent system, the Rank-based system effectively caps his fan vote advantage. Our simulation shows he would have faced elimination risks in the finals, potentially altering the championship outcome to a more technically proficient couple.
\end{itemize}

\begin{figure}[htbp]
    \centering
    % Row 1: The two Bristol Palin cases
    \begin{minipage}{0.48\textwidth}
        \centering
        \includegraphics[width=\linewidth]{model2_imgs/Task2_Subtask2_Timeline_Bristol_Palin_S11.png}
        \subcaption{Bristol Palin (S11)}
    \end{minipage}\hfill
    \begin{minipage}{0.48\textwidth}
        \centering
        \includegraphics[width=\linewidth]{model2_imgs/Task2_Subtask2_Timeline_Bristol_Palin_S15.png}
        \subcaption{Bristol Palin (S15)}
    \end{minipage}

    \vspace{0.5cm} % Spacing between rows

    % Row 2: The other three cases (Jerry, Billy, Bobby)
    \begin{minipage}{0.32\textwidth}
        \centering
        \includegraphics[width=\linewidth]{model2_imgs/Task2_Subtask2_Timeline_Jerry_Rice_S2.png}
        \subcaption{Jerry Rice (S2)}
    \end{minipage}\hfill
    \begin{minipage}{0.32\textwidth}
        \centering
        \includegraphics[width=\linewidth]{model2_imgs/Task2_Subtask2_Timeline_Billy_Ray_Cyrus_S4.png}
        \subcaption{Billy Ray Cyrus (S4)}
    \end{minipage}\hfill
    \begin{minipage}{0.32\textwidth}
        \centering
        \includegraphics[width=\linewidth]{model2_imgs/Task2_Subtask2_Timeline_Bobby_Bones_S27.png}
        \subcaption{Bobby Bones (S27)}
    \end{minipage}

    \caption{Combined Counterfactual Survival Timelines. The Red blocks ("Eliminated") appearing in the "Rank + Save" rows (and Rank Rule) indicate where the historical outcome of "Safe" (Green) would have been overturned. This demonstrates the proposed rule's ability to filter out low-scoring survivors earlier in the competition.}
    \label{fig:combined_timelines}
\end{figure}

\subsection{Multi-Objective Evaluation System}
We construct a tri-dimensional metric system to evaluate the overall quality of a competition format:

\begin{enumerate}
    \item \textbf{Fairness Index ($I_{fair}$):} Spearman correlation between Final Rank and Judge Rank. Measures professional integrity.
          \begin{equation}
              I_{fair} = \rho(\mathbf{R}_{final}, \mathbf{R}_{judge})
          \end{equation}
    \item \textbf{Fan Satisfaction Index ($I_{fan}$):} Spearman correlation between Final Rank and Fan Rank. Measures entertainment value.
          \begin{equation}
              I_{fan} = \rho(\mathbf{R}_{final}, \mathbf{R}_{fan})
          \end{equation}
    \item \textbf{Extreme Risk Rate ($R_{risk}$):} The probability of a "System Failure", defined as the elimination of the absolute best dancer (Judge Rank 1) or the absolute crowd favorite (Fan Rank 1).
\end{enumerate}

\subsection{Policy Recommendation}
We propose a weighted composite score $S(\alpha)$ to simulate different policy preferences, where $\alpha$ is the weight assigned to Fan Satisfaction:
\begin{equation}
    S(\alpha) = (1 - \alpha) \cdot I_{fair} + \alpha \cdot I_{fan} - \lambda \cdot R_{risk}
\end{equation}

\begin{figure}[H]
    \centering
    \includegraphics[width=0.8\linewidth]{model2_imgs/Task2_Subtask3_Sensitivity.png}
    \caption{Sensitivity Analysis of Competition Formats. The "Rank + Save" system (Orange Line) maintains the highest composite score across the "Balanced Zone" ($0.4 < \alpha < 0.6$), proving it is the most robust compromise between fairness and popularity.}
    \label{fig:sensitivity}
\end{figure}

\begin{figure}[H]
    \centering
    \includegraphics[width=0.8\linewidth]{model2_imgs/Task2_Subtask3_Risk.png}
    \caption{Extreme Risk Analysis. The "Rank + Save" mechanism has the lowest probability of eliminating the best dancer (Professional Collapse), while Percent-based systems carry a significantly higher risk of such anomalies.}
    \label{fig:risk}
\end{figure}

\textbf{Conclusion & Recommendation:}
\begin{itemize}
    \item The \textbf{Rank-Based System} is superior for competitive equity. It normalizes the high variance of fan votes, preventing a single viral star from breaking the game mechanics.
    \item The \textbf{Judges' Save} is an essential safety valve. Our simulations show it reduces the $R_{risk}$ (elimination of talent) by 35\% without significantly harming fan satisfaction.
\end{itemize}

\textbf{Final Verdict:} We strongly recommend the adoption of the \textbf{Rank-Based System with Judges' Save} (the current S28+ format) as the Pareto-optimal solution for future seasons.


\section{Model III: Dynamic Comparison and Optimization of the Combination Systems}

\subsection{Problem Formulation: The Dual-Track Simulation Framework}
Having reconstructed the latent fan voting distribution $\hat{\mathbf{S}}_t$
in Model I, we proceed to conducting
an evaluation of the competition's combination methods.
Throughout the thirty-four seasons of DWTS,
the mechanism for combining professional evaluations (Judges)
and public popularity (Fans) has shifted between two distinct paradigms:
\begin{itemize}
    \item \textbf{Rank-based System:} Employed in Seasons 1-2 and 28+,
          utilizing the sum of ordinal ranks.
          This system inherently equalizes the variance between the two components.
    \item \textbf{Percentage-based System:} Employed in Seasons 3-27,
          utilizing the sum of normalized cardinal shares.
          This system is sensitive to the magnitude of vote distribution.
\end{itemize}

Our objective is to quantify the intrinsic "Fan-Friendliness" of these systems
via a counterfactual simulation.
We establish a \textbf{Dual-Track Simulation} framework:
for every historical week $t$,
we fix the contestants' Judge Scores $\mathbf{J}_t$ and
Estimated Fan Votes $\hat{\mathbf{S}}_t$ (derived from Model I),
while varying the combination logic to generate parallel outcomes.

\subsubsection{Standardization and Metric Definition}
To facilitate a rigorous comparison between the ordinal nature of ranks
and the cardinal nature of percentages,
we introduce the \textbf{Fan Deviation Index (FDI)}.
This metric quantifies the distortion imposed by the combination rule
upon the pure will of the audience.

First, we convert the estimated fan voting share $\mathbf{P}_{F,t}$ into rankings:
\begin{equation}
    \mathbf{R}_{F,t} = \text{Rank}(-\mathbf{P}_{F,t})
\end{equation}

The FDI for a given rule $k \in \{\text{Rank}, \text{Percent}\}$ is defined as the
\textbf{normalized Manhattan distance} between the rule-derived final ranking $\mathbf{R}_{\text{final}, k}$ and the pure fan ranking $\mathbf{R}_{F}$:
\begin{equation}
    \text{FDI}_k^{(t)} = \frac{1}{N_t} \sum_{i=1}^{N_t} \left| R_{\text{final}, k, i} - R_{F, i} \right|
\end{equation}
where $N_t$ is the number of contestants in week $t$.

Explanation:
\begin{itemize}
    \item Low FDI ($\to 0$): The system faithfully preserves the ordinal preferences of the audience (High Fan-Friendliness).
    \item High FDI: The system allows judge scores to significantly override fan preferences (Judge-Dominant).
\end{itemize}

\subsection{Analysis Results Comparison}
Our analysis of FDI across 34 seasons reveals a consistent bias,
as illustrated in Figure \ref{fig:fdi_boxplot} and \ref{fig:fdi_barchart}.
The Percent-based rule consistently exhibits lower FDI values compared to the Rank-based rule.

Defining the differential $\Delta = \text{FDI}_{\text{Rank}} - \text{FDI}_{\text{Percent}}$,
we observe that $\Delta > 0$ in over \textbf{83\%} of simulated weeks. This implies that the \textbf{Percentage-based System is inherently more Fan-Friendly}.
Positive $\Delta$ values indicate the Percent-based system yields results closer to the fan vote.

\begin{figure}[H]
    \centering
    \begin{minipage}{0.48\textwidth}
        \centering
        \includegraphics[width=\linewidth]{model3_imgs/Task2_Subtask1_Boxplot.png}
        \caption{Distribution Box Plot of FDI}
        \label{fig:fdi_boxplot}
    \end{minipage}\hfill
    \begin{minipage}{0.48\textwidth}
        \centering
        \includegraphics[width=\linewidth]{model3_imgs/Task2_Subtask1_BarChart.png}
        \caption{Weekly Comparison of FDI}
        \label{fig:fdi_barchart}
    \end{minipage}
\end{figure}


This phenomenon is attributable to the mismatch of variance.
Judge scores are mostly constrained to a narrow range (typically 7-9),
corresponding to a low level of variation.
In contrast, fan votes follow a heavy-tailed distribution
(e.g., a viral star capturing 40\% of the vote while others receive single digits).
In a linear summation $P_J + P_F$,
the component with higher variance—the fan vote—mathematically dominates the outcome.

Conversely, \textbf{the Rank system} forces a uniform distribution onto both components,
effectively suppressing the "superstar effect" and restoring equal power to the judges.

\subsection{Mechanism Analysis}
To evaluate the robustness of these systems against extreme anomalies,
we examine historical "Low-Score Survivors"—contestants who advanced
despite poor technical performance, often sparking controversy~\cite{wade2014dancing}.
We selected four representative cases:
Jerry Rice (S2), Billy Ray Cyrus (S4), Bristol Palin (S11,S15), and Bobby Bones (S27).
By simulating their performance under four distinct combination rules (Rank/Percent w/wo Judges' Choice), we analyze how different mechanisms impact their survival.

As illustrated in Figure \ref{fig:combined_timelines},
different mechanisms exhibit varying results for these contestants:

\begin{itemize}
    \item \textbf{Percentage-based System:} Often allows candidates with massive fan bases (e.g., Bobby Bones) to neutralize low judge scores, securing their advancement in the competition.
    \item \textbf{Rank-based System:} Compresses the "superstar" effect of fan votes, making it harder for popularity to override technical deficiencies.
    \item \textbf{Rank + Judges' Choice:} This combination proves to be the most rigorous filter.
          In our simulation, controversial contestants like Bristol Palin and Billy Ray Cyrus would have been eliminated significantly earlier than their actual departure.
\end{itemize}

\begin{figure}[H]
    \centering
    \includegraphics[width=1\linewidth]{model3_imgs/Task2_Subtask2_Timeline_2plus3.png}
    \caption{Survival Timelines of Controversial Contestants}
    \label{fig:combined_timelines}
\end{figure}

Our analysis demonstrates that the \textbf{Rank-based System with Judges' Choice} can effectively
screen out controversial contestants with insufficient dance skills and
improve overall ~\cite{dehnart2010audience}.
The "Choice" mechanism provides a final safeguard,
allowing judges to elect the superior dancer
when technically weaker contestants land in the bottom two,
thus preventing technical outliers from advancing solely on popularity.

\subsection{Policy Optimization: A Multi-Objective Evaluation}
Concluding our analysis, we propose a comprehensive evaluation framework to guide future rule selection. We model the rule selection as a multi-objective optimization problem that balances two conflicting values: \textbf{Technical Fairness} (Meritocracy) and \textbf{Fan Satisfaction} (Popularity).

\subsubsection{Metric Definitions}
To quantify these abstract concepts, we define three key metrics based on the correlation between simulation results and ground-truth preferences:
\begin{enumerate}
    \item \textbf{Fairness Index ($I_{\text{fair}}$):}
          Defined as the Spearman rank correlation coefficient ($\rho$) between the final rank derived from the combination rule and the pure \textit{Judge Rank}.
          \begin{equation}
          I_{\text{fair}} = \rho(\mathbf{R}_{\text{final}}, \mathbf{R}_{\text{judge}}) = 1 - \frac{6 \sum_{i=1}^{N_t} (R_{\text{final}, i} - R_{\text{judge}, i})^2}{N_t(N_t^2 - 1)}
          \end{equation}
          where $R_{\text{final}, i}$ and $R_{\text{judge}, i}$ represent the final rank and technical rank of contestant $i$ respectively, and $N_t$ is the number of contestants in week $t$.
          A value close to 1 implies the system strictly adheres to professional technical evaluation.

    \item \textbf{Fan Satisfaction Index ($I_{\text{fan}}$):}
          Similarly, defined as the Spearman rank correlation between the final rank and the pure \textit{Fan Rank}.
          \begin{equation}
          I_{\text{fan}} = \rho(\mathbf{R}_{\text{final}}, \mathbf{R}_{\text{fan}})
          \end{equation}
          A higher value indicates the outcomes strongly reflect audience preferences.

    \item \textbf{Extreme Risk Rate ($R_{\text{risk}}$):}
          Defined as the accidental elimination of the
          strongest contenders,
          which measures how often the \textit{Judge's No.1} or \textit{Fan's No.1} is eliminated.
\end{enumerate}

\subsubsection{The Composite Utility Function and Alpha ($\alpha$)}
Since fairness and fan satisfaction are often inversely related, we introduce a preference parameter $\alpha \in [0, 1]$
to represent the show producers' strategic trade-off:
\begin{equation}
    S(\alpha) = (1 - \alpha) \cdot I_{\text{fair}} + \alpha \cdot I_{\text{fan}}
\end{equation}
The parameter $\alpha$ could describe the show's identity:
\begin{itemize}
    \item $\alpha \to 0 / 1$: Prioritizes Technical skills (0) / Popularity (1).
    \item $\alpha \in [0.4, 0.6]$: The "Balanced Zone", is ideal for a commercial TV show that requires both high performance quality and viewer engagement~\cite{goldderby2019democracy}.
\end{itemize}

\begin{figure}[H]
    \centering
    \begin{minipage}{0.48\textwidth}
        \centering
        \includegraphics[width=\linewidth]{model3_imgs/Task2_Subtask3_Sensitivity.png}
        \caption{Sensitivity Analysis of Competition Formats}
        \label{fig:sensitivity_analysis_0}
    \end{minipage}\hfill
    \begin{minipage}{0.48\textwidth}
        \centering
        \includegraphics[width=\linewidth]{model3_imgs/Task2_Subtask3_Risk.png}
        \caption{Extreme Risk Analysis}
        \label{fig:risk_analysis}
    \end{minipage}
\end{figure}

From the sensitivity analysis (Figure \ref{fig:sensitivity_analysis}), we observe that
the ``Rank'' system scores higher than ``Percent'' system in the ``Balanced Zone'' ($0.4 < \alpha < 0.6$).
The ``Rank+Choice'' mechanism demonstrates higher probability of eliminating the top-ranked dancer by judges (Figure \ref{fig:risk_analysis}),which marks a better ability to limit the extreme number of votes from the audience.
\subsubsection{Policy Recommendation}

Our analysis (Figure \ref{fig:sensitivity_analysis} \& Figure \ref{fig:risk_analysis}) demonstrates that the \textbf{Rank-based System with Judges' Choice} is the Pareto-optimal solution.
\begin{itemize}
    \item \textbf{Why Rank over Percent?} While the Percent system maximizes $I_{\text{fan}}$, it incurs an unacceptably high $R_{\text{risk}}$ of eliminating talented professionals (Figure \ref{fig:risk_analysis}). The Rank system provides structural equity.
    \item \textbf{Why Include Judges' Choice?} It acts as an essential safety valve.It can help us reduce the probability of the audience's autocratic power and reduce the probability of dancers with extremely high audience votes but extremely low judges entering the finals.
\end{itemize}

\textbf{Final Verdict:} We strongly recommend retaining the current \textbf{Rank-based System with Judges' Choice} (S28+ format) for future seasons. It effectively prevents the "Controversial Survivor" phenomenon while maintaining high engagement, satisfying the dual goals of fairness and entertainment.


\section{Model IV: The Adaptive Logistic Meritocracy System}

\subsection{Problem Formulation and Motivation}
Our analysis mentioned above exposed a critical dichotomy in the existing aggregation rules:
\begin{itemize}
\item \textbf{Rank Rule (The Elitist Trap):} While ensuring high fairness by curbing the variance of fan votes, it suffers from "Structural Insensitivity." The reduction of vote magnitudes to ordinal ranks alienates the audience, leading to historically low satisfaction.
\item \textbf{Percent Rule (The Populist Trap):} While maximizing fan satisfaction, it is plagued by "Linear Vulnerability." The lack of a saturation mechanism means that a single "popularity monster" (e.g., Bobby Bones in Season 27) can mathematically nullify the judges' input, rendering professional evaluation irrelevant.
\end{itemize}

To resolve this dilemma, we propose a "Third Way": a mechanism that respects the magnitude of the popular vote while imposing a soft ceiling on monopoly. We term this the \textbf{Adaptive Logistic Meritocracy System}.

\subsection{Mathematical Model Establishment}

\subsubsection{Sigmoid Transformation: The "Soft Ceiling" Mechanism}
The core innovation of our model is the replacement of the linear vote share  with a non-linear effective score  via a Sigmoid transformation:
\begin{equation}
S(x) = \frac{1}{1 + e^{-K \cdot (x - c)}}
\end{equation}
where:
\begin{itemize}
\item $\mathrm{x}$: The share of raw fan votes of a contestant.
\item $\mathrm{c}$: The "Average Line" (center), representing the baseline performance expectation.
\item $\mathrm{K}$: The slope parameter that controls the sensitivity of the system.
\end{itemize}

\textbf{Mechanism Analysis:}
As illustrated in Figure \ref{fig:sigmoid_mechanism}, this transformation creates three distinct zones:
\begin{itemize}
\item \textbf{Penalty Zone:} Contestants underperforming the average face score suppression, discouraging "free-riding."
\item \textbf{Incentive Zone:} Near the average, the curve is steepest. This high marginal return (which is maximized) incentivizes fierce competition among the mid-tier contestants, where every vote counts significantly.
\item \textbf{Saturation Zone:} For extreme outliers, the curve flattens (Diminishing Marginal Utility). This acts as a "Circuit Breaker," ensuring that even if a celebrity garners 50\% of the vote, their effective score is capped, preserving the relevance of judge scores.
\end{itemize}

\begin{figure}[H]
\centering
\includegraphics[width=0.8\textwidth]{Task4_Mechanism_Adaptive.png}
\caption{Mechanism of Adaptive Meritocracy. The system dynamically adjusts its slope  based on competition intensity.}
\label{fig:sigmoid_mechanism}
\end{figure}

\subsubsection{Dynamic Adaptation: The "Smart Gain" Control}
A static cannot accommodate the varying dynamics of different weeks. We introduce a dynamic adjustment mechanism where  is inversely proportional to the standard deviation of the vote distribution:
\begin{equation}
K_{week} = K_{base} + \frac{\alpha}{\sigma_{votes}}
\end{equation}
This effectively creates an \textbf{Adaptive Gain Control}:
\begin{itemize}
\item \textbf{Tight Race:} When vote shares are clustered,  spikes. The system becomes highly sensitive, functioning like a "microscope" to differentiate close competitors.
\item \textbf{Blowout:} When a monopoly exists, relaxes to its base value, focusing primarily on the saturation effect to cap the leader.
\end{itemize}

\subsubsection{Normalization and Synthesis}
To integrate this non-linear score with the linear judge scores, we employ a "One-vs-Rest" normalization strategy:
\begin{equation}
FinalScore_i = 0.5 \times P_{Judge, i} + 0.5 \times \frac{S(x_i)}{\sum_j S(x_j)}
\end{equation}
This ensures the structural integrity of the 50-50 weighting mandated by the competition format.

\subsection{Simulation and Validation}

\subsubsection{Case Study: The "Correction" of Bobby Bones}
We stress-tested the model using Season 27, the "anomaly season" dominated by Bobby Bones. Figure \ref{fig:bobby_impact} visualizes his trajectory under the old Percent Rule versus our Adaptive System.

\begin{itemize}
\item \textbf{Observation:} Under the original rules (Purple Dashed Line), Bones consistently ranked \#1 despite poor dancing. Under the Adaptive System (Blue Solid Line), his effective rank drops significantly in multiple weeks (e.g., -2 rank drop in Week 4).
\item \textbf{Result:} The system successfully identified his vote share as "inflationary" and applied the saturation mechanism. Although he remains competitive due to immense popularity, he is no longer invincible, forcing a reliance on dance improvement to survive.
\end{itemize}

\begin{figure}[H]
\centering
\includegraphics[width=0.8\textwidth]{Task4_Impact_Bobby_Bones_Refined.png}
\caption{Impact Analysis on Bobby Bones (S27)  (Note: Week 5 was a non-elimination week and is excluded).}
\label{fig:bobby_impact}
\end{figure}

\subsection{Comparative Evaluation}
We conducted a comprehensive comparison between the Rank Rule, Percent Rule, and our Adaptive System across all 34 seasons. Figure \ref{fig:three_way_metrics} presents the results.

\begin{figure}[H]
\centering
\includegraphics[width=0.8\textwidth]{Task4_Metrics_Final_ThreeWay.png}
\caption{Comprehensive Metric Comparison}
\label{fig:three_way_metrics}
\end{figure}

\textbf{Quantitative Results:}
\begin{itemize}
\item \textbf{Fairness Index($I_{\text{fair}}$) vs. Percent Rule:} The Adaptive System increases the correlation with judge rankings by \textbf{13.3\%}. This confirms the restoration of professional integrity.
\item \textbf{Fan Satisfaction Index($I_{\text{fan}}$) vs. Rank Rule:} Compared to the elitist Rank Rule, our system preserves fan satisfaction by \textbf{18.0\%}.
\item \textbf{The Trade-off:} We achieve this massive gain in fairness at the cost of only a marginal \textbf{4.6\%} drop in fan satisfaction compared to the populist Percent Rule.
\end{itemize}

\subsection{Conclusion}
The simulation confirms that the \textbf{Adaptive Logistic Meritocracy System} is not merely a compromise, but a robust optimization. By implementing \textbf{Non-linear Saturation} and \textbf{Dynamic Sensitivity}, it successfully constructs a "Trade-off Frontier" that outperforms the binary choice of previous eras. It is the optimal theoretical model for the future of \textit{Dancing with the Stars}, balancing the dual imperatives of professional credibility and mass entertainment.


\section{Model Details and Prospect}

\subsection{Sensitivity Analysis \& Parameter Details}

To validate the robustness of our models, we conducted a systematic sensitivity analysis, focusing primarily on Model I (Bayesian Estimation), as it serves as the foundational data source for all subsequent analyses. The defined parameter $\alpha$ in Model III \textbf{has been examined} according to figure \ref{fig:sensitivity_analysis_0}.

\subsubsection{Parameter Sensitivity in Model I}

The reliability of our inferred fan votes depends on two key assumptions: $\lambda=0.4$ and $\kappa=50$ to control the original distribution of the vote shares.

To test whether our results are reliable enough, we performed a \textbf{Grid Search Experiment} across a broad parameter space:

$\lambda \in [0.0, 0.5]$ with step length of 0.1;$\qquad\qquad\kappa \in [10, 90]$, step length 20.

For each pair $(\lambda, \kappa)$, we re-ran the inference process across representative seasons and calculated the \textbf{Historical Reproduction Rate} and the \textbf{Average CV Value} to compare consistency and certainty.

\begin{figure}[H]
    \centering
    \begin{minipage}{0.48\textwidth}
        \centering
        \includegraphics[width=\linewidth]{figures/sensitivity_accuracy.png}
        \label{fig:sens_xy_accuracy}
    \end{minipage}
    \hfill
    \begin{minipage}{0.48\textwidth}
        \centering
        \includegraphics[width=\linewidth]{figures/sensitivity_cv.png}
        \label{fig:sens_xy_cv}
    \end{minipage}
    \caption{Sensitivity Analysis of Model I across ($\lambda$, $\kappa$) Space}
\end{figure}
Our analysis yields two key insights regarding accuracy and uncertainty:

For accuracy (Left), the 3D surface plot reveals a broad "high plateau," where the historical reproduction rate remains consistently above \textbf{98\%} across a wide range of parameter combinations. This indicates that our model is \textbf{highly robust} and \textbf{insensitive} to specific choices of $\lambda$ and $\kappa$, proving that the estimation validity is driven by the data structure rather than parameter fine-tuning.

Regarding the regulation of uncertainty (Right), while accuracy is stable, the uncertainty (CV) shows a clear gradient driven by $\kappa$. This is mathematically expected, as $\kappa$ controls the variance of the Dirichlet distribution. We selected $\kappa=50$ not to maximize accuracy but to maintain a realistic level of uncertainty ($CV \approx 0.43$) that reflects the inherent ambiguity of fan voting.

In conclusion, our choice of $\lambda=0.4$ and $\kappa=50$ places the model in a high-accuracy region while maintaining a realistic level of uncertainty, effectively balancing predictive power with statistical honesty.


\subsubsection{Parameter Optimization in Model IV}

These heatmaps visualize the impact of Base Slope ($K_{base}$) and Sensitivity ($\alpha$) on model performance, comparatively showing Fairness ($I_{\text{fair}}$), Satisfaction ($I_{\text{fan}}$), and the Composite Score. The selected parameter set $(K_{base}=5, \alpha=10)$, highlighted by the blue box, resides in the optimal "sweet spot," maximizing the composite objective by balancing the trade-off between competitive differentiation and mass appeal.

\begin{figure}[H]
    \centering
    \includegraphics[width=1.0\textwidth]{Task4_Parameter_Sensitivity.png}
    \caption{Parameter Details for the Adaptive System}
    \label{fig:sensitivity_analysis}
\end{figure}

\subsection{Future Work}

\subsubsection{Model Extension: Towards Game-Theoretic Robustness}

While our Adaptive Logistic Meritocracy System successfully addresses the fairness-satisfaction trade-off, future research should scrutinize its robustness through \textbf{Algorithmic Game Theory}. A critical extension is to verify \textbf{Incentive Compatibility (IC)} against strategic voting. Although we assume sincere voting, some fan groups might engage in dishonest voting and manipulate outcomes. Future studies, ideally using individual-level ballot data, could focus on modeling stakeholders as rational agents to determine whether the system admits a \textit{Nash Equilibrium} in which truthful voting remains the dominant strategy. A primary bottleneck for these economic extensions is data granularity. Our current model relies on aggregated vote shares ($\hat{\mathbf{S}}_t$). To rigorously test the excellent economic effects, we require \textbf{individual-level ballot data} (e.g., identifying if user $i$ voted for contestant $A$ implies a zero probability of voting for rival $B$). 

\subsubsection{Model Application: Beyond the Show}

The core architecture of our model—a dual-track aggregation system with non-linear saturation—has broad applicability beyond reality television. It offers a generalizable framework for any domain requiring the integration of \textbf{"Expert Wisdom"} and \textbf{"Crowd Preference"}. For instance, our dynamic sensitivity parameter ($K$) could serve as an automated quality control filter: detecting anomalous variance in community votes and automatically increasing the weight of expert reviewers to stabilize the outcome. Moreover, we envision deploying this model as a background "Shadow Scorer" for show producers. It would generate live metrics on "Fairness Risk," alerting producers when the divergence between popularity and skill exceeds a safety threshold, potentially triggering automatic interventions.

\newpage
\section{Memo}

\noindent
\textbf{To:} The Producers of \textit{Dancing with the Stars} \\
\textbf{From:} MCM Team 2617432 \\
\textbf{Date:} February 2, 2026 \\
\textbf{Subject:} Suggestions for Improving the Rating System

\par\noindent\rule{\textwidth}{0.2pt}

\noindent Dear Producers,

\noindent
\begin{minipage}[t]{0.65\textwidth}
    \setlength{\parindent}{2em}
    \indent We've analyzed 34 seasons of data from \textit{Dancing with the Stars}, trying to understand how different rating combination rules affect the competition results. Now, we've come to some conclusions and suggestions that we believe can help make the show even better in the future.
\end{minipage}
\hfill
\begin{minipage}[t]{0.3\textwidth} % 右侧占据 30% 的宽度
    \vspace{-1cm} % 稍微向上微调图片位置
    \raggedleft % 图片靠右对齐
    \includegraphics[width=3cm]{logo_new.png}
\end{minipage}


\subsubsection*{What We Found from Past Rules}

We compared the two main systems you have used:

\begin{itemize}
    \item \textbf{Percentage System:} This method clearly reflects fan passion but carries high risk. Extremely popular contestants can easily overpower judge scores, making dancing skill less important.
    \item \textbf{Rank System:} This method is fairer but partially ignores fans' support. A lead of one million votes counts the same as a single vote, which will dampen fan enthusiasm.
    \item \textbf{Judges' Choice:} Allowing judges to save one of the bottom two couples is highly effective. It prevents talented dancers from being eliminated due to voting anomalies.
\end{itemize}

\subsubsection*{Our Proposal: The Adaptive System}

To seek for better solutione, we designed the \textbf{Adaptive Logistic Meritocracy System} (ALMS) to balance fairness and popularity without compromise.

\textbf{1. Limiting Extreme Monopolies:}
The system counts fan votes fully up to a reasonable point, then gradually reduces the weight of excess votes. This prevents a single contestant from breaking the game mechanics while still rewarding popularity.

\textbf{2. Dynamic Sensitivity:}
The system automatically adjusts based on competition tightness. It becomes more sensitive to small differences when the race is close to ensure the best dancer wins.

Finally, our tests show that this system improves fairness by 13.3\% compared to the Percentage rule while retaining 95.4\% of fan satisfaction. We suggest using this model for future seasons, for it truly maintains the excitement of voting while protecting the competitive nature of the show.

Besides, retaining the Judges' Save is a good idea to ensure technical integrity. You should also focus on "the Middle": Voting impact is highest for mid-tier contestants. Highlighting these close battles will drive more meaningful engagement than focusing on safe leaders.

We hope our analysis provides a fresh and useful perspective for your team.


\newpage
% All references are managed in the BibTeX file `references.bib`.
% Use \cite{} in the main text, for example: ``... as shown in~\cite{Krawitz2025}.''
\phantomsection
\addcontentsline{toc}{section}{References}
\bibliographystyle{plain}
\bibliography{references}

\label{LastPage}

\newpage
\AImatter
\begin{ReportAiUse}{9}
\bibitem{1}

Baidu Fanyi, Baidu Translate (Sep 10, 2025 version) \\
Uploaded entire paper writen in Mandarin to be translated into English. 

\bibitem{2}

GitHub CoPilot (Jan 16, 2024 version)\\
Auto-completions for code used in preparing our models.

\bibitem{3}

Bing AI\\   %这里写AI的类型
Query1: \\ %这里写你的问题
Output:     %这里写ai的回复

\bibitem{4}

Bing AI\\   %这里写AI的类型
Query1: \\ %这里写你的问题
Output:     %这里写ai的回复

\end{ReportAiUse}

\end{document}

